\documentclass[11pt,a4paper,twoside]{article}
\usepackage[utf8]{inputenc}
\usepackage{polski}



\usepackage{graphicx}
\usepackage{subcaption} % obrazki obok siebie
%\usepackage[caption=false]{subfig} % obrazki nad sobą
\usepackage{wrapfig} 
\usepackage{float}
\usepackage{geometry}
%\geometry{lmargin=3cm,rmargin=2cm, tmargin=2.5cm, bmargin=2.5cm} %marginesy w pracy inż/mgr
\geometry{lmargin=2.5cm,rmargin=2.5cm, tmargin=2.5cm, bmargin=2.5cm} %marginesy ogólne
\usepackage{multirow} % scalanie wierszy tabeli

\usepackage{fancyhdr}
\pagestyle{fancy}
\fancyhead{} % wszystkie nagłówki puste
%\fancyhead[RE,LO]{ Absolwenci Wydziału Prawa  2012}
\fancyfoot{} % wszystkie stopki puste
\fancyfoot[LE,RO]{\thepage}
\renewcommand{\headrulewidth}{0pt}
%\renewcommand{\footrulewidth}{0.4pt}

\usepackage{hyperref}% nazwy odsyłaczy

%unikanie myślników dzielących słowa między liniami
\tolerance=1
\emergencystretch=\maxdimen
\hyphenpenalty=10000
\hbadness=10000

\usepackage{algorithm}
\usepackage{algorithmicx}
\usepackage{algpseudocode}
\makeatletter
\renewcommand{\ALG@name}{Algorytm}
\renewcommand{\figurename}{Wykres}

\usepackage{enumitem}
\setitemize{itemsep=2pt,topsep=2pt,parsep=2pt,partopsep=2pt} %odstępy wyliczanych elementów (-)

\usepackage{indentfirst} % wcięcie w pierwszym akapicie (pod)rozdziału
%%%%%%%%%%%%%%%%%%%%%%%%%%%%%%%%%%%%%%%%% formatowanie kodu R
\usepackage{listings} 
\usepackage{color}
\definecolor{gray}{rgb}{0.5,0.5,0.5}
\lstset{
language=R,
basicstyle=\footnotesize\ttfamily,%\scriptsize\ttfamily,
commentstyle=\ttfamily\color{gray},
numbers=left,
numberstyle=\ttfamily\color{gray}\footnotesize,
stepnumber=0, % numeracja linii
numbersep=5pt,
backgroundcolor=\color{white},
showspaces=false,
showstringspaces=false,
showtabs=false,
frame=none, % obramowanie kodu
tabsize=2,
captionpos=b,
breaklines=true,
breakatwhitespace=false,
title=\lstname,
escapeinside={},
keywordstyle={},
morekeywords={}
}
%%%%%%%%%%%%%%%%%%%%%%%%%%%%%%%%%%%%%%%%%

%%% zawijanie tekstu w tabelach zgodnie z życzeniem
\usepackage{stackengine}
\usepackage{array}
\newcolumntype{L}[1]{>{\raggedright\arraybackslash}p{#1}}
\setstackEOL{\#}
\setstackgap{L}{12pt}
%%%

\usepackage{amsfonts} % zbiory liczba (np. naturalnych)
\usepackage{amsmath} %duże klamry
\usepackage{bbm} %skok jednostkowy
\usepackage[titletoc,title]{appendix} % dodatki - zmiana wyświetlania nagłówka
\pagenumbering{gobble}
\usepackage{afterpage} % pusta strona
\usepackage{tabularx}

%\usepackage{setspace} % interlinia
%\singlespacing 
%\onehalfspacing
%\doublespacing
%\setstretch{0.96}

\begin{document}

\begin{center}
\vspace*{3\baselineskip}
{\LARGE{GIS Projekt}}
\\
\vspace*{1\baselineskip}
{\large{Problem komiwojażera w obecności ulic jednokierunkowych.}}
\\
\vspace*{1\baselineskip}
Tomasz Korzeniowski, 265753\\
Jacek Sochacki, 259741
\\
\vspace*{1\baselineskip}
\today
\end{center}
\section{Zadanie}
Należy zaimplementować (nietrywialny) algorytm rozwiązujący problem komiwojażera, a następnie zbadać, czy można modelować ulice jednokierunkowe w ten sposób, że połączeniu w kierunku A$\rightarrow$B (A, B - wybrane miasta) nadaje się wagę równą odległości miedzy miastami, a połączeniu B$\rightarrow$A nadaje się wagę o bardzo dużej wartości. Głównym celem zadania jest identyfikacja warunków, w których algorytm wskazuje drogę „pod prąd” pomimo istnienia drogi „legalnej”.
\section{Opis zadania}
\subsection{Problem komiwojażera}
Problem komiwojażera można przedstawić następująco. Rozważmy graf $G=(V, E)$, gdzie $V$ to zbiór $n$ miast, które mają zostać odwiedzone przez komiwojażera, natomiast $E$ to zbiór możliwych połączeń między tymi miastami. W ogólności rozważamy graf pełny. Potrzebujemy wyznaczyć macierz kosztów przejść między miastami, czyli wagi krawędzi $c_{ij}$ między miastem $i$ oraz $j$. Do zaznaczenia, którymi krawędziami przejedzie komiwojażer wykorzystamy zmienne binarne $x_{ij}$
$$x_{ij} = 
	\begin{cases} 
      1, & \text{gdy komiwojażer przejeżdża z miasta i do j} \\
      0, & \text{w przeciwnym przypaku}\\
   \end{cases}
$$

Wiadomo, że każde miasto musi być odwiedzone dokładnie raz, z wyjątkiem miasta startowego, w którym komiwojażer zakończy swoją podróż. Sprowadza się to do znalezienia drogi w grafie $G$, której pierwszy i ostatni wierzchołek są tożsame, a koszt przejazdu wyznaczoną drogą jest jak najmniejszy. 

Powyższe zadanie można opisać następującym modelem

\begin{equation}
\min \sum_{i,j \in V} c_{ij} x_{ij} 
\label{f_celu}
\end{equation}
\begin{equation}
\sum_{j \in V} x_{ij}=1, \quad i \in V \backslash \{j\} 
\end{equation}
\begin{equation}
\sum_{i \in V} x_{ij}=1, \quad j \in V \backslash \{i\} 
\end{equation}
\begin{equation}
x_{ij} \in \{0,1\}
\end{equation}
\begin{equation}
x \text{ jest drogą w grafie G}
\end{equation}

%Zauważmy, że model znajduje drogę, a nie cykl. Nie jest to jednak przeszkodą, jeśli założymy, że znalezioną drogę zamkniemy w cykl przez dodanie krawędzi między pierwszym i ostatnim wierzchołkiem.

%Obliczenie kosztu podróży sprowadza się wtedy do wzoru $\sum_{i=1}^{n-1}c_{i(i+1)} + c_{n1}$, gdzie $i$ jest indeksem kolejnego wierzchołka.

Zauważmy, że w macierzy $X = [x_{ij}]_{n\times n}$ w każdej kolumnie i każdym wierszu występuje dokładnie jedna jedynka. Zatem można reprezentować znaleziony cykl jako wektor $w$ kolejnych wierzchołków odwiedzanych przez komiwojażera. Wiemy, że każde miasto występuje w tym wektorze tylko raz, a cykl zamykany jest przez skrajne wierzchołki wektora. Tak utworzony cykl zawsze będzie cyklem Hamiltona, jeśli operacjami wykonywanymi na jego elementach będzie zamiana ich kolejności.

Tak przedstawione zadanie rozwiązuje problem komiwojażera dla grafu pełnego. Zadanie jednak nie jest jedynie problemem komiwojażera. Należy rozważyć połączenia jednokierunkowe między dowolnymi miastami \mbox{$A$ i $B$}, zatem musimy rozważać graf skierowany. Struktura problemu i przyjęty model nie musi być zmieniany. Gwarantuje nam to odpowiednia konstrukcja macierzy kosztów $C = [c_{ij}]_{n\times n}$. W przeciwieństwie do grafu nieskierowanego nie będzie ona symetryczna. Zatem otrzymujemy graf skierowany gdzie ulice jednokierunkowe będą reprezentowane przez krawędzie o nieskończonym koszcie. Ze względu na trudności w implementacji nieskończoności ograniczymy się do dostatecznie dużej wartości takiej krawędzi.

W ramach zadania sprawdzimy czy powyższe modelowanie ulic jednokierunkowych jest wystarczające do znalezienia dopuszczalnego rozwiązania problemu komiwojażera. Spróbujemy wskazać warunki w jakich algorytm znajduje błędne rozwiązania (ulice „pod prąd”).

\subsection{Przeszukiwanie tabu}
Wybranym sposobem rozwiązania problemu będzie algorytm przeszukiwania tabu. Jest to metaheurystyka stosowana do rozwiązywania problemów optymalizacyjnych. Polega na  wykonaniu ciągu kolejnych ruchów w celu przeszukiwania przestrzeni rozwiązań dopuszczalnych zadania. 

Przeszukiwanie tabu rozpoczyna się wybraniem rozwiązania początkowego. W każdej iteracji znajdowane jest rozwiązanie lokalnie najlepsze, a także poszukiwani są sąsiedzi (w zbiorze rozwiązań dopuszczalnych) takiego rozwiązania. Sąsiadem rozwiązania $w$ nazywamy ciąg wierzchołków $w'$, który różni się od oryginalnego na dokładnie dwóch pozycjach. Najlepszy ze wszystkich znalezionych sąsiadów staje się nowym rozwiązaniem lokalnym dla kolejnej iteracji. Najlepszy z sąsiadów charakteryzuje się mniejszą wartością funkcji celu (\ref{f_celu}) od obecnego rozwiązania.

Wybrany algorytm bierze swoją nazwę z ruchów tabu. Jest to sposób zaznaczania, które wierzchołki zostały zamienione. Na listę tabu wpisujemy liczbę iteracji algorytmu, przez którą ponowna zamiana tychże wierzchołków staje się zakazana (tabu). Z każdą iteracją algorytmu czas, przez który wcześniej zazanczony ruch jest zakazany, zostaje zmniejszony o 1. Lista tabu jest listą tylko z nazwy. W rzeczywistości będziemy operowali na macierzy górnej trójkątnej $LT = [lt_{ij}]_{n\times n}$. Element $lt_{ij}$ będzie zawierał informację o liczbie iteracji, przez które zamiana wierzchołków $i$ oraz $j$ jest zakazana. Gdy wartość $lt_{ij} = 0$, dany ruch przestanie być zakazany. 

W implementacji, powyższa dekrementacja listy będzie mało wydajna. Z tego względu będziemy na liście tabu zapisywać aktualny numer iteracji algorytmu powiękoszony o stałą liczbę iteracji przez które dany ruch jest tabu. Do sprawdzenia czy dany ruch jest zakazany, będziemy porównywać wartość listy tabu tego ruchu z akrulnym licznikiem iteracji. Gdy wartość listy tabu będzie co najmniej taka jak aktualy licznik iteracji, ruch przestanie być zakazany. Trudno określić jak długo elementy listy tabu powinny być zakazane. Autorzy \cite{FogelHeurystyka} sugerują by czas pobytu na liście tabu był równy $3n$, gdzie $n$ oznacza liczbę miast problemu.

Na początku działania algorytmu potrzebne jest pewne wstępne rozwiązanie. W naszym przypadku będzie to różnowartościowy ciąg, zawierający wszystkie liczby naturalne ze zbioru $[1,n]$. Rozważamy grafy pełne, więc wybór ulicy jednokierunkowej nie jest przeszkodą. Krawędzie które „nie istnieją” (nie ma połączenia między danymi miastami) lub są to ulice jednokierunkowe mają po prostu bardzo duży koszt.

\subsection{2-opt move}
Poszukiwanie sąsiada $w'$ obecnego rozwiązania wykorzystuje technikę „2-opt move”, Metoda ta została przewidziana dla grafów nieskierowanych. Polega na zamianie dwóch wierzchołków w poszukiwanej drodze, które nie sąsiadują. Zmiana ta powoduje usunięcie dwóch krawędzi z oryginalnej drogi, zamianie wierzchołków i ponowne połączenie przestawionych wierzchołków w drogę. Dla przykładu rozważmy drogę $w$ o następujących wierzchołkach $v$

\bgroup
\def\arraystretch{1.5}
\begin{tabular}{cccccccccccc}
%\hline
Przed zmianą: & $\cdots$ & $v_{i-1}$ & $v_{i}$ & $\mathbf{v_{i+1}}$ & $v_{i+2}$ & $\cdots$ & $v_{j-1}$ & $\mathbf{v_{j}}$ & $v_{j+1}$ & $v_{i+2}$ & $\cdots$\\
Po zmianie: & $\cdots$ & $v_{i-1}$ & $v_{i}$ & $\mathbf{v_{j}}$ & $v_{i+2}$ & $\cdots$ & $v_{j-1}$ & $\mathbf{v_{i+1}}$ & $v_{j+1}$ & $v_{i+2}$ & $\cdots$\\
\end{tabular}
\egroup

Zostały usunięte krawędzie między wierzchołkami $(v_{i}, v_{i+1})$ oraz $(v_{j}, v_{j+1})$, lecz powstały dwie nowe krawędzie $(v_{i}, v_{j})$ oraz $(v_{i+1}, v_{j+1})$. Dla reszty krawędzi pozostała droga przed zmianą, a dla niektórych z nich zmienił się kierunek ich przechodzenia w nowo wyznaczonej drodze. Zamianę tę prezentuje wykres \ref{2opt4krawedzie}a.

Ze względu na skierowanie grafu naszego zadania, technika „2-opt move” zostanie zachowana, lecz zamiast dwóch krawędzi, zostaną zmienione cztery. Wynika to z wykresu \ref{2opt4krawedzie}b. Po dodaniu skierowania cyklu, widać, że nie da się dodać dwóch nowych krawędzi tak, aby pozostałe krawędzie można było przejść zgodnie z ich kierunkiem przed zmianą.

\begin{figure}[ht]
\vspace{-0pt}
\centering
\includegraphics[width=10cm]{2opt4krawedzie}
\caption{Zamiana wierzchołków $v_{i+1}$ oraz $v_{j}$ zgodnie z 2-opt move dla grafu nieskierowanego (a) oraz skierowanego (b).}
\label{2opt4krawedzie}
\end{figure}



Przyjmując zamianę wierzchołków jak w przykładzie grafu nieskierowanego, zostaną usunięte krawędzie $(v_{i}, v_{i+1})$, $(v_{j}, v_{j+1})$ jak poprzednio oraz dodatkowo $(v_{i+1}, v_{i+2})$ i $(v_{j-1}, v_{j})$. Zamiast nich powstaną krawędzie $(v_{i}, v_{j})$, $(v_{i+1}, v_{j+1})$ jak wcześniej oraz nowe $(v_{j}, v_{j+2})$ oraz $(v_{j-1}, v_{i+1})$. Graficznie zamianę tę przedstawia wykres \ref{2optMove}.

\begin{figure}[ht]
\vspace{-0pt}
\centering
\includegraphics[width=15cm]{2optMove}
\caption{Zamiana wierzchołków $v_{i+1}$ oraz $v_{j}$ zgodnie z 2-opt move dla grafu skierowanego.}
\label{2optMove}
\end{figure}

Taki sposób zamiany niesąsiadujących wierzchołków umożliwia znalezienie wszystkich sąsiadów obecnego rozwiązania. Wystarczy, że sprawdzimy każdą parę wierzchołków, nawet tych sąsiadujących. Sąsiad o najmniejszej wartości funkcji celu stanie się wtedy bieżącym rozwiązaniem. Do obliczenia wartości funkcji celu wystarczy przejść nowo wyznaczonym cyklem, sumując wagi kolejnych krawędzi.

Ponadto, w celu skrócenia czasu wykonania algorytmu wprowadzimy licznik $count$. Będzie on odpowiedzialny za zliczanie iteracji, w których znaleziony sąsiad nie daje lepszego rozwiązania od aktualnie znanego.

\subsection{Kryterium aspiracji}
Może się zdarzyć, że jeden z rozważanych sąsiadów objętych tabu daje świetny wynik, lepszy od jakiegokolwiek rozważanego wcześniej rozwiązania. W standardowym podejściu, algorytm przeszukiwania, wybrałby najlepsze z rozwiązań, w których nie ma ruchów zakazanych. Takie podejście powodowałoby, że algorytm jest mało elastyczny. Jeśli sytuacja jest nietypowa (znaleziono najlepsze ze wszystkich dotąd znanych rozwiązań) możemy zapomnieć o zakazie i wybierzemy nietypowe rozwiązanie. 

\subsection{Schemat algorytmu}
Schemat działania przeszukiwania tabu przestawia algorytm \ref{tabuSearchAlg}. Licznik $tries$ odpowiada za liczbę prób poszukiwania rozwiązania. W ramach każdej próby zostanie wykonanych $iter$ poszukiwań sąsiada i aktualizacji listy tabu. Najlepsze rozwiązanie w danej próbie (lokalne) jest przechowywane w $bestSolverScore$. Jeśli to rozwiązanie jest najlepsze z dotychczas znalezionych dla wszystkich prób (globalne) zostanie zapamiętane w $bestSolutionScore$. Jeśli jednak przez $count$ iteracji algorytmu rozwiązanie nie poprawi się względem dotychczas najlepszego znalezionego, nie ma sensu dalej poszukiwać. Może to wynikać z osiągnięcia minimum lokalnego w przestrzeni rozwiązań dopuszczalnych.

\begin{algorithm}[ht]
\caption{Przeszukiwanie tabu}
\label{tabuSearchAlg}
\begin{algorithmic}%[1]
\Require $C$
	\State $tries$ $\leftarrow$ 0
	\While{$tries$ $\neq$ MAX-TRIES}
		\State wygeneruj drogę początkową $w$
		\State $count$ $\leftarrow$ 0
		\State $iter$ $\leftarrow$ 0
		\While{$iter$ $\neq$ MAX-ITER}
		
		\While{istnieje nieodwiedzony sąsiad $w$} %\Comment{znajdź wszystkich sąsiadów drogi $w$}
		\If{$w'$ ma mniejszą wartość funkcji celu od innych sąsiadów nieobjętych tabu \textbf{LUB} $w'$ spełnia kryterium aspiracji}
		%\State zapamiętaj $w'$ jako najlepszego sąsiada 
		\State wybierz najlepszego sąsiada: $w \leftarrow w'$
		\EndIf
		\EndWhile
			%\State znajdź wszystkich sąsiadów drogi $w$: $w'$% i wybierz najlepszego z nich: $w'$
			%\State zamień odpowiednie wierzchołki: $w \leftarrow w'$ 
			
			
%			\If{$w$ jest lepsza dla danego $tries$}
%				\State $bestSolverScore \leftarrow w$
%				\State $count$ $\leftarrow$ 0
%				\If{$w$ jest najlepsza dla wszystkich $tries$}
%					\State $bestSolutionScore \leftarrow w$
					
%    			\EndIf
			\State uaktualnij listę tabu: $lt_{ij} \leftarrow iter + 3n$
    		\If{$w'$ jest lepszy od $w$ dla danego $tries$}
				\State $bestSolverScore \leftarrow w$
				\State $count$ $\leftarrow$ 0
				\If{$w$ jest najlepszy dla wszystkich $tries$}
					\State $bestSolutionScore \leftarrow w$
    			\EndIf
    		\Else
    			\State ++$count$
    		\EndIf
    		\If{$count =$ MAX-COUNT}
    			\State zbyt długo nie znaleziono lepszego rozwiązania - przerwij tę próbę
    		\EndIf
    	\State ++$iter$
		\EndWhile
		\State ++$tries$
	\EndWhile
\end{algorithmic}
\end{algorithm}



\section{Implementacja algorytmu}

\subsection{Struktury danych}
Do wyznaczenia rozwiązania problemu komiwojażera będziemy potrzebowali
\bgroup
\def\arraystretch{1.5}
\begin{tabularx}{\textwidth}{l|X}
%\hline
$C = [c_{ij}]_{n\times n}$ & Macierz kosztów przejść między $n$ miastami. Zakładamy, że graf jest pełny, a ulice kierunkowe przyjmują wartość wystarczająco dużą, żeby uznać ją za nieskończoność (np. 1e38). Reprezentacją takiej macierzy będzie tablica dwuwymiarowa (tablica tablic) typu zmiennoprzecinkowego podwójnej precyzji (double).\\
$w$ & Wektor reprezentujący drogę komiwojażera. Zawiera indeksy kolejnych odwiedzonych miast, zatem będzie typu całkowitoliczbowego. \\
$LT = [lt_{ij}]_{n\times n}$ & Lista tabu - macierz zakazanych przejść. Element $lt_{ij}$ przechowuje informację o liczbie iteracji przez które zamiana miast $i$ oraz $j$ jest zakazana. Typ całkowitoliczbowy.\\
\end{tabularx}
\egroup

Ponadto będą potrzebne iteratory pętli czy zmienna przechowująca wartość funkcji celu.

\subsection{Projekty testów}
W ramach testów należy sprawdzić poprawność działania algorytmu. Początkowo będzie dany graf pełny, dla którego znany jest koszt drogi optymalnej. Liczba wierzchołków grafu będzie rzędu kilkunastu. Program kilkukrotnie poszuka rozwiązania, a następnie najlepsze z nich zostanie zapamiętane i porównane z rozwiązaniem optymalnym.

Drugim krokiem będzie zakazanie ruchu w określonych połączeniach, które nie należały do najlepszego rozwiązania. Koszt niedozwolonych połączeń będzie zwiększony o umowną wartość nieskończoności - 1e38. Tak dobrana wartość może powodować utratę precyzji, lecz nie jest to problem, gdyż interesuje nas liczba ulic jednokierunkowych w znalezionym rozwiązaniu a nie dokładna wartość funkcji celu rozwiązania. Testy te zostaną poprowadzone dla kilku określonych procentów połączeń jednokierunkowych.
	
Następnie zostaną wygenerowanie losowe macierze kosztów $C$ wraz z zadanym procentem ulic jednokierunkowych. Wagi krawędzi zostaną dostosowane tak, by zapewnić istnienie cyklu Hamiltona w grafie. Pozwoli to zbadać zadania dużej skali (ok. 1000 wierzchołków).
	
Wszystkie te testy zostaną poprowadzone dla różnej liczby miast. Wyniki zostaną przedstawione w postaci tabeli. Dla określonej liczby miast i procentu niedozwolonych połączeń zostanie przedstawiony procent prawidłowych rozwiązań, a dla grafów pełnych podany zostanie również stosunek rozwiązania optymalnego do najlepszego odnalezionego przez algorytm. 
\subsection{Założenia programu}
Implementacja przeszukiwania tabu zostanie wykonana w języku C++. Program będzie wyświetlał na standardowym wyjściu znaleziony cykl komiwojażera oraz jego koszt. Dane o problemie do rozwiązania będą pobierane z pliku tekstowego. W pierwszej linii tego pliku zapisana będzie liczba miast, a w kolejnych koszty przejścia między odpowiednimi nimi, oddzielone spacjami.

\section{Wyniki}
Dla każdego z przyjętych procentów ulic jednokierunkowych oraz liczby miast wykonano 10 testów. Testy dla mniejszych grafów (10-250 wierzchołków) wykonywały 5 prób poszukiwania rozwiązania ($tries=5$). Testy na dużych grafach umożliwiały jedynie jedną próbę. Maksymalna liczba iteracji algorytmu, którą algorytm mógł wykonać to MAX-ITER = 10000. Jeżeli przez MAX-COUNT = 200 iteracji wynik nie uległ poprawie, obliczenia zostawały przerywane i rozpoczynała się kolejna próba testu. Dane testowe zostały wygenerowane w taki sposób, żeby zawsze istniało przynajmniej jedno poprawne rozwiązanie problemu komiwojażera.

Za poprawne wykonanie algorytmu uznawane jest znalezienie cyklu Hamiltona, w którym nie występuje żadna ulica jednokierunkowa. Procentowy udział sukcesów w wykonanych testach przedstawia tabela \ref{sukcesy}. Wartości ,,--" oznaczają, że nie da się uzyskać danych spełniających warunki na istnienie cyklu oraz zadanego procenta ulic jednokierunkowych. 

Można zaobserwować, że jeśli ulice jednokierunkowe stanowią co najwyżej 30\% wszystkich dostępnych ulic, algorytm przeszukiwania tabu zawsze znajduje poprawne rozwiązanie. Ponadto, wraz ze wzrostem liczby wierzchołków granica procentowego udziału ulic jednokierunkowych wzrasta. Dopiero po przekroczeniu 90\% algorytm nie znajduje poprawnego rozwiązania.

\bgroup
\def\arraystretch{1.2}
\begin{table}[ht]
\centering
\begin{tabular}{|c|c|c|c|c|c|c|c|}
\hline
 & \multicolumn{7}{c|}{Procent ulic jednokierunkowych}\\\hline
Liczba miast & 0 & 10 & 30 & 50 & 70& 80& 90\\  \hline
10& 100& 100& 100& 30& 20& --& --\\ \hline
25& 100& 100& 100& 90& 0& 0& --\\ \hline
50& 100& 100& 100& 100& 20& 0& 0\\ \hline
100& 100& 100& 100& 100& 60& 0& 0\\ \hline
250& 100& 100& 100& 100& 90& 0& 0\\ \hline
500& 100& 100& 100& 100& 90& 40& 0\\ \hline
1000& 100& 100& 100& 100& 100& 80& 0\\ \hline
\end{tabular}
\caption{Procent sukcesów.}
\label{sukcesy}
\end{table}
\egroup

\bgroup
\def\arraystretch{1.2}
\begin{table}[ht]
\centering
\begin{tabular}{|c|c|c|c|c|c|c|c|}
\hline
 & \multicolumn{7}{c|}{Procent ulic jednokierunkowych}\\\hline
Liczba miast & 0 & 10 & 30 & 50 & 70& 80& 90\\  \hline
10& 210& 208& 207& 205& 205& --& --\\ \hline
25& 236& 234& 229& 223& 214& 212& --\\ \hline
50& 303& 296& 280& 255& 233& 233& 230\\ \hline
100& 436& 432& 203& 352& 283& 273& 268\\ \hline
250& 954& 927& 804& 693& 511& 386& 394\\ \hline
500& 2042& 1890& 1686& 1392& 861& 645& 610\\ \hline
1000& 4424& 4183& 3577& 2883& 2095& 1441& 1040\\ \hline
\end{tabular}
\caption{Średnia liczba iteracji.}
\label{liczbaIteracji}
\end{table}
\egroup

W tabeli \ref{liczbaIteracji} zostały przedstawione średnie liczby iteracji algorytmu. Dla bardzo małych grafów (10-25 wierzchołków) liczba iteracji jest niemal stała. Dla większych grafów wraz ze wzrostem liczby ulic zakazanych, liczba iteracji maleje. Wynika to z faktu, że każda zamiana wierzchołków w cyklu wiąże się z możliwością dodania krawędzi zakazanej. Im większa liczba takich krawędzi tym większe szanse na gorsze rozwiązanie, zwiększenie licznika $count$ i wcześniejsze zakończenie poszukiwań. Powyższą sytuację najlepiej widać w przypadku grafów o 500 i 1000 wierzchołków.

%Uzasadnieniem takiej obserwacji jest fakt, że każda zamiana wierzchołków w cyklu może zmniejszyć jego koszt przez usuwanie krawędzi, lecz jednocześnie dodawane krawędzie drastycznie go podnoszą. W takim przypadku, rozważany sąsiad nie zostanie zapamiętany. Jeśli sytuacja powtórzy się wielokrotnie, doprowadzi to do zwiększenia licznika $count$ i wcześniejszego zakończenia poszukiwania. 

Ponadto czas zakazu $lt_{ij}$ długości $3n$ jest słuszny dla małych grafów. W przypadku dużych grafów wydaje się zbyt długi. Dla grafu o 1000 wierzchołków ruch tabu będzie zakazany przez 3000 iteracji. Szukany cykl liczy 1000 krawędzi, czyli po 1000 iteracji wiele z możliwych ruchów jest zakazanych przez około 2000 kolejnych iteracji. Tak długi czas oczekiwania na nowego sąsiada może zostać złamany jedynie przez kryterium aspiracji, które to zachodzi dość rzadko. Algorytm byłby mało wydajny, gdyby pozwolić mu tak długo czekać na kolejne rozwiązanie, więc licznik braku poprawy rozwiązania $count$ przerywa poszukiwania.

\bgroup
\def\arraystretch{1.2}
\begin{table}[ht]
\centering
\begin{tabular}{|c|c|c|c|c|c|c|c|}
\hline
 & \multicolumn{7}{c|}{Procent ulic jednokierunkowych}\\\hline
Liczba miast & 0 & 10 & 30 & 50 & 70& 80& 90\\  \hline
10& 0,005& 0,004& 0,003& 0,002& 0,002& --& --\\ \hline
25& 0,046& 0,05& 0,044& 0,082& 0,04& 0,04& --\\ \hline
50& 0,5& 0,48& 0,44& 0,38& 0,35& 0,37& 0,37\\ \hline
100& 4,9& 4,8& 4,44& 3,9& 3,05& 3,04& 2,9\\ \hline
250& 165,15& 151,98& 129,47& 106,5& 76,1& 53,12& 54,81 \\ \hline
500& 402,8& 405,9& 360,5& 284,2& 185,8& 117,6& 111,43\\ \hline
1000& 8,8h& 8h& 6,1h& 5,5h& 4,3h& 2,7h& 2h\\ \hline
\end{tabular}
\caption{Średni czas wykonania algorytmu.}
\label{czasWykonania}
\end{table}
\egroup

W tabeli \ref{czasWykonania} zawarto średnie czasy wykonania algorytmu. Wyniki wyrażone są w sekundach, o ile nie podano inaczej. Na ich podstawie można łatwo zauważyć, że czas wykonania algorytmu nie jest wprost proporcjonalny do liczby wykonanych iteracji. 

Ciekawym zjawiskiem jest wartość odchylenia standardowego czasu wykonania, zebranego w tabeli \ref{odchylenieCzasuWykonania}. Im większy procent ulic jednokierunkowych tym mniejsze odchylenie standardowe.

\bgroup
\def\arraystretch{1.2}
\begin{table}[ht]
\centering
\begin{tabular}{|c|c|c|c|c|c|c|c|}
\hline
 & \multicolumn{7}{c|}{Procent ulic jednokierunkowych}\\\hline
Liczba miast & 0 & 10 & 30 & 50 & 70& 80& 90\\  \hline
10& 0,003& 0& 0& 0& 0,001& --& --\\ \hline
25& 0,006& 0,004& 0,007& 0,12& 0,006& 0,006& --\\ \hline
50& 0,035& 0,066& 0,053& 0,032& 0,036& 0,076& 0,086\\ \hline
100& 0,3& 0,2& 0,26& 0,17& 0,17& 0,13& 0,2\\ \hline
250& 14,3& 11,08& 7,5& 7,8& 6,67& 4,32& 4,64\\ \hline
500& 66,7& 16,03& 20,9& 13,3& 31,94& 18,6& 7,78\\ \hline
1000& 1,5h& 1,3h& 1,3h& 0,9h& 0,6h& 0,4h& 4min\\ \hline
\end{tabular}
\caption{Odchylenie standardowe czasu wykonania algorytmu.}
\label{odchylenieCzasuWykonania}
\end{table}
\egroup

W tabeli \ref{sredniafCelu} przedstawiono średnią wartość funkcji celu znalezionych rozwiązań. Wraz ze wzrostem liczby ulic jednokierunkowych w grafie wartość funkcji celu drastycznie rośnie. Dane testowe zostały przygotowane tak by znane było rozwiązanie optymalne. Rozwiązaniem tym jest wartość funkcji celu równa liczbie wierzchołków grafu. Oznacza to, że zawsze istnieje cykl Hamiltona, którego każda krawędź ma wagę równą 1. Dzięki temu można zauważyć, że wraz ze wzrostem liczby wierzchołków grafu, otrzymane rozwiązanie jest coraz bliższe optymalnemu. Zależność ta nie dotyczy przypadków, w których nie znaleziono poprawnego rozwiązania (90\% ulic jednokierunkowych). Można także łatwo oszacować liczbę ulic jednokierunkowych znalezionych przez algorytm w przypadku 70\% i 80\% udziału ulic jednokierunkowych.

\bgroup
\def\arraystretch{1.2}
\begin{table}[ht]
\centering
\begin{tabular}{|c|c|c|c|c|c|c|c|}
\hline
 & \multicolumn{7}{c|}{Procent ulic jednokierunkowych}\\\hline
Liczba miast & 0 & 10 & 30 & 50 & 70& 80& 90\\  \hline
10& 115,7& 159& 162,2& 81& 10& --& --\\ \hline
25& 256& 290& 396,9& 538,73& 3e+38& 7e+38& --\\ \hline
50& 422,3& 461,6& 629,3& 1057,4& 1883& 7e+38& 1,3e+39\\ \hline
100& 751,6& 797,25& 1052& 1644,3& 3381,4& 5e+38& 1,9e+39\\ \hline
250& 1486,8& 1587& 2103& 3118& 5664,6& 3,6e+38& 2,21e+39\\ \hline
500& 2362& 2678& 3512,1& 5009,9& 9177,91& 15883,8& 2,36e+39\\ \hline
1000& 3959,3& 4221,2& 5760,1& 8291,9& 14231& 24660& 2,21e+39\\ \hline
\end{tabular}
\caption{Średnia wartość funkcji celu.}
\label{sredniafCelu}
\end{table}
\egroup

Na podstawie tabeli \ref{odchylenieSredniafCelu} widać, że wraz ze wzrostem udziału ulic jednokierunkowych rośnie odchylenie standardowe. Świadczy to większej dokładności znalezionego wyniku dla mniejszych procentów ulic jednokierunkowych w danych.

\bgroup
\def\arraystretch{1.2}
\begin{table}[ht]
\centering
\begin{tabular}{|c|c|c|c|c|c|c|c|}
\hline
 & \multicolumn{7}{c|}{Procent ulic jednokierunkowych}\\\hline
Liczba miast & 0 & 10 & 30 & 50 & 70& 80& 90\\  \hline
10& 42,31& 64,46& 84,03& 84,75& 0& --& --\\ \hline
25& 44,61& 58,9& 81,1& 142,8& 1,3e+38& 1,3e+38& --\\ \hline
50& 52,07& 68,2& 55,2& 149,7& 157,07& 3,2e+38& 1,2e+38\\ \hline
100& 75,27& 60,02& 81,9& 166,5& 220& 2,5e+38& 2,3e+38\\ \hline
250& 85& 99,3& 146,1& 145,6& 277,5& 1,6e+38& 5,6e+38\\ \hline
500& 81,8& 117& 101,6& 181,78& 459,43& 926,84& 2,01e+38\\ \hline
1000& 60,22& 88,2& 182,7& 219,3& 443,7& 727,6& 3e+38\\ \hline
\end{tabular}
\caption{Odchylenie standardowe wartości funkcji celu.}
\label{odchylenieSredniafCelu}
\end{table}
\egroup

Powyższe wyniki wskazują, że warunkami do wskazania błędnego rozwiązania przez algorytm przeszukiwania tabu jest odpowiednio duży udział ulic jednokierunkowych w danych.

Do wykonania testów wykorzystano komputer z 64-bitowym systemem operacyjnym Windows 7, procesorze Intel Core i5-3210M 2.5 GHz oraz pamięci operacyjnej 8 GB. 

%\newpage
  
%\clearpage
\begin{thebibliography}{9}
\addcontentsline{toc}{section}{Literatura}
\bibitem{FogelHeurystyka}
Z. Michalewicz, D. B. Fogel
\emph{Jak to rozwiązać czyli nowoczesna heurystyka}.
WNT, Warszawa 2006
\end{thebibliography}

\end{document}